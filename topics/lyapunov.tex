
\begin{sstBreakBox}[Plum]{{
				\Large
				Lyapunov
			}\small
		% Александр Михайлович Ляпунов
	}

	\begin{sstOnlyFrame}[Plum]
		\textbf{Stability} is a property of an
		\textbf{equilibrium point} $\bar{\mathbf{x}}$
		of a system
	\end{sstOnlyFrame}

	\begin{sstOnlyFrame}[Plum]
		\begin{definition}[Lyapunov Stability]
			$\bar{\mathbf{x}}$ is \textbf{Lyapunov stable} if:

			$\forall\ \epsilon>0\ \exists\ \delta(\epsilon)$ s.t.
			$| x(0) - \bar{x} |_{\scriptscriptstyle 2}< \delta(\epsilon) \to
				| x(k) - \bar{x}|_{\scriptscriptstyle 2} < \epsilon$
		\end{definition}
	\end{sstOnlyFrame}

	\begin{sstOnlyFrame}[Plum]
		\begin{definition}[Globally asymptotic stability]
			If $\bar{\mathbf{x}}$ is attractive, i.e.,
			$\lim_{k\to\infty} ||x(k)-\bar{x}||=0,\ \forall x(0)$
			and Lyapunov stable
			then $\bar{\mathbf{x}}$ is \textbf{globally asymptotically stable}.
		\end{definition}
	\end{sstOnlyFrame}

	\begin{sstOnlyFrame}[Plum]
		\begin{definition}[Global Lyapunov function]
			For the equilibrium $\bar{\mathbf{x}}=0$
			of a system $x(k+1)=g(x(k))$,
			a function $V$, continuous at the origin,
			finite and such that

			$\forall\ x\in \mathbb{R}^{n}$:
			\[
				|x|   \to\infty              \Rightarrow V(x)  \to\infty\]
			\[
				V(x)=0 \ \text{ if }\ x=0 \quad\text{else}\quad V(x)>0             \]
			\[
				V(g(x)) - V(x) \leq -\alpha(x)                          \]
			\[
				\text{for continuous positive definite }
				\alpha:\mathbb{R}^n\to \mathbb{R}
			\]

			then  $V:\mathbb{R}^n\to \mathbb{R}$
			is called \textbf{Lyapunov function}.
		\end{definition}
	\end{sstOnlyFrame}

	\begin{sstOnlyFrame}[Plum]
		\begin{theorem}
			%TODO: non linear system (1)
			If a system admits a Lyapunov function $V(x)$,
			then $\bar{\mathbf{x}} = 0$ is
			\textbf{globally asymptotically stable}.
		\end{theorem}
	\end{sstOnlyFrame}

	\begin{sstOnlyFrame}[Plum]
		\begin{theorem}[Lyapunov indirect method]
			System linearized around $\bar{\mathbf{x}}=0$
			with resulting matrix $A$ and eigenvalues $\lambda_i$.

			If	$\forall |\lambda_i| < 1$
			then the origin is asymptotically stable.

			if $\exists |\lambda_i| > 1$
			then origin is unstable.

			If $\exists |\lambda_i| = 1$
			we can't conclude anything about stability.
		\end{theorem}
	\end{sstOnlyFrame}

	\begin{sstOnlyFrame}[Plum]
		\begin{minipage}[c]{0.34\linewidth}
			\textbf{Discrete-Time}

			\textbf{Lyapunov equation}
		\end{minipage}
		\begin{minipage}[b]{0.64\linewidth}
			\begin{equation}
				A^TPA-P=-Q,\quad Q>0
				\label{lipunov:dt}
			\end{equation}
		\end{minipage}
	\end{sstOnlyFrame}

	\begin{sstOnlyFrame}[Plum]
		\begin{theorem}[Existence of solution, DT Lyapunov equation]
			The discrete-time Lyapunov equation has a unique solution
			$P > 0$
			% iff $A$ has all eigenvalues inside the unit circle, i.e.
			iff the system $x(k+1) = Ax(k)$ is stable.
		\end{theorem}
	\end{sstOnlyFrame}
\end{sstBreakBox}
%TODO: Lyapunov for LQR to Lyapunov L2.p43
