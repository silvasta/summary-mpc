

\begin{definition}[Positively Invariant Set $\mathcal{O}$]
	For an autonomous or closed-loop system,
	the set $\mathcal{O}$ is positively invariant if:
	\[
		x(k)\in\mathcal{O}\Rightarrow
		x(k+1) \in \mathcal{O},
		\quad \forall k \in \{0,1,\dots\}
	\]
\end{definition}

\begin{definition}[Maximal Positively Invariant Set $\mathcal{O}_\infty$]
	A set that contains all $\mathcal{O}$
	is the maximal positively invariant set
	$\mathcal{O}_\infty \subset \mathcal{X}$
\end{definition}

\begin{definition}[Pre-Sets]
	The set of states that
	in the dynamic system $x(k+1) = g(x(k))$
	in one time step evolves into the target set $\mathcal{S}$
	is the \textbf{pre-set} of $\mathcal{S}$
	$\Rightarrow \pre(\mathcal{S}) := \{x\mid g(x)\in \mathcal{S}\}$
\end{definition}

\begin{theorem}[Geometric condition for invariance]
	Set $\mathcal{O}$ is positively invariant set iff
	$\mathcal{O}\subseteq\pre(\mathcal{O})
		\Leftrightarrow
		\pre(\mathcal{O})\cap\mathcal{O} = \mathcal{O}$
\end{theorem}
\begin{proof}
	\textbf{Necessary} if
	$\mathcal{O} \nsubseteq  \mathrm{pre}(\mathcal{O})$,
	then $\exists\bar{x} \in \mathcal{O}$
	s.t $\bar{x} \notin \mathrm{pre}(\mathcal{O})$
	$\leadsto \bar{x}\in\mathcal{O},
		\bar{x}\notin\mathrm{pre}(\mathcal{O})$,
	thus $\mathcal{O}$ not positively invariant

	\textbf{Sufficient} if
	$\mathcal{O}$ not pos invar set,
	then $\exists \bar{x}\in\mathcal{O}$
	s.t $g(\bar{x}) \notin\mathcal{O}$
	$\leadsto \bar{x}\in\mathcal{O},
		\bar{x}\notin\mathrm{pre}(\mathcal{O})$
	thus $\mathcal{O}\notin \mathrm{pre}(\mathcal{O})$
\end{proof}

\begin{lemma}{\ssthl{Invariant Sets from Lyapunov Functions}}

	If $V:\mathbb{R}^n \to \mathbb{R}$ is a Lyapunov function
	for $x(k+1) = g(x(k))$, then
	$Y := \{x \mid V(x) \leq \alpha\}$
	is an invariant set for all $\alpha \geq 0$
\end{lemma}
\begin{proof}
	Lyapunov property $V(g(x)) - V(x) < 0$
	implies that once $V(x(k))\leq \alpha$,
	$V(x(j))<\alpha$,
	$\forall\ j\ge k \rightarrow$ Invariance
\end{proof}

\textbf{Example System}
$x(k+1) = Ax(k)$, $A^\top P A - P \prec 0 \prec P$
and resulting  Lyapunov function
$V(x(k)) = x(k)^\top P x(k)$

\textbf{Goal} Find the largest $\alpha$ s.t the
invarinat set $Y_\alpha \in \mathcal{X}$

$\qquad Y_\alpha := \{x \mid x^\top P x \leq \alpha\}\subset \mathcal{X} := \{x \mid Fx\leq f\}$

Equivalent to $\max_\alpha \alpha \quad \text{s.t. }\
	h_{Y_\alpha}(F_i) \leq f_i \ \forall i \in \{1\dots n\}$

%TODO: Maximum Elipsoid set

\subsection{Control Invariance}


\begin{definition}[Control Invariant Set]
	$\mathcal{C} \subseteq \mathcal{X}$ control invariant if
	$$x(k) \in \mathcal{C} \Rightarrow
		\ \exists u(k) \in \mathcal{U} \text{ s.t }
		g(x(k),u(k))\in\mathcal{C} \ \forall k$$
\end{definition}

\begin{definition}[Maximal Control Invariant Set $\mathcal{C}_\infty$]
	A set that contains all $\mathcal{C}$
	is the maximal positively invariant set
	$\mathcal{C}_\infty \subset \mathcal{X}$

	\textbf{Intuition} For all states in $\mathcal{C}_\infty$
	exists control law s.t constraints are never violated
	$\leadsto$ \textbf{The best any controller could ever do}
\end{definition}

\textbf{Pre-set}
$\pre(\mathcal{S}):=\{x \mid \exists u\in\mathcal{U}
	\text{ s.t } g(x,u) \in \mathcal{S}\}$

Set $\mathcal{C}$ is control invariant iff:
$\mathcal{C}\subseteq\pre(\mathcal{C})
	\Leftrightarrow
	\pre(\mathcal{C})\cap\mathcal{C} = \mathcal{C}$

\begin{sstTitleBox}[BrickRed]{
		Control Law from Control Invariant Set
	}

	\begin{sstOnlyFrame}[BrickRed]

		\begin{centering}
			Control law $\kappa(x(k))$ will \textbf{guarantee}
			that the system
			with control invariant set $\mathcal{C}$
			satisfies constraints \textbf{for all time} if
			$$x(k+1)=g(x(k),u(k))\to
				g(x,\kappa(x)) \in \mathcal{C} \ \forall x \in \mathcal{C}$$

			We can use this fact to \textbf{synthesize}
			control law $\kappa$

		\end{centering}
		\begin{sstFullFrame}[BrickRed]\color{white}
			\[
				\kappa(x):=\argmin\{f(x,u) \mid g(x,u)\in\mathcal{C}\}
			\]
		\end{sstFullFrame}

		with $f$ as any function (including $f(x,u)=0)$
	\end{sstOnlyFrame}

	\begin{sstOnlyFrame}[BrickRed]
		Does not ensure that system will converge

		Difficult because calculating control invariant sets is hard

		\textbf{MPC} implicitly describes $\mathcal{C}$
		s.t easy to represent/compute
	\end{sstOnlyFrame}
\end{sstTitleBox}


\subsection{Computing Invariant Sets and Pre-sets}

%WARN: Structure compute invariant set

\begin{minipage}{0.53\linewidth}
	\begin{sstFullFrame}[ForestGreen]
		\color{white}
		\begin{algorithmic}
			\State $\Omega_0 \leftarrow \mathcal{X}$
			\Loop
			\State $\Omega_{i+1} \leftarrow \mathrm{pre}(\Omega_i)\cap\Omega_i$
			\If{$\Omega_{i+1}=\Omega_i$}
			\State\Return $\mathcal{O}_\infty = \Omega_i$
			\EndIf
			\EndLoop
		\end{algorithmic}
	\end{sstFullFrame}
\end{minipage}

(Same but much harder for control invariat sets)


\begin{align*}
	\intertext{\textbf{System for Pre-Set Computation}}
	x(k+1)                & = Ax(k) +Bu(k)
	% & \text{dynamics}
	\\
	u(k)\in\mathcal{U}  : & =\{u|Gu\le g\}
	% & \text{constraints}
	\\
	\mathcal{S}:          & = \{x|Fx\leq f\}
	% & \text{set to check}
	\intertext{\textbf{Invariant Pre-Set}}
	\pre(S):              & =  \{x \mid Ax \in S\}                               \\
	                      & =  \{ x \mid FAx \leq f\}                            \\
	\intertext{\textbf{Control Invariant Pre-Set}}
	\pre(S):              & =  \{x \mid\exists u\in\mathcal{U}, Ax+Bu \in S\}    \\
	                      & =  \{x \mid\exists u\in\mathcal{U}, FAx+FBu \leq f\} \\
	                      & =  \left\{x \mid\exists u\in\mathcal{U},
	\begin{bmatrix} FA & FB \\0&G \end{bmatrix}
	\begin{bmatrix} x \\u \end{bmatrix}
	\leq \begin{bmatrix} f \\g \end{bmatrix} \right\}
	\intertext{This is a \textbf{projection} operation}
\end{align*}

