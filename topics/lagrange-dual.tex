
\subsection{Optimality Conditions}

\begin{sstTitleBox}{
		Lagrange Duality
	}
	\begin{centering}
		\vspace{-1.5mm}
		\begin{equation}\text{Consider }
			f^\star = \inf_{x\in\mathcal{\mathbb{R}}^n}f(x)
			\text{ s.t.}\ g(x)\le0,h(x)=0
			\label{eq:dual}
		\end{equation}

		%TODO: Definitions lambda...

		\begin{sstFullFrame}
			{\color{white}
				\vspace{-2mm}
				\[\begin{aligned}
						 & \textbf{Lagrangian}
						 & \hspace{-4mm}	\mathcal{L}(x,\lambda,\nu)
						 & = f(x) + \lambda\T g(x)+\nu\T h(x)
						\\
						\\
						 & \textbf{Dual Function}
						 & \hspace{-4mm}		d(\lambda,\nu)            & = \inf_{x \in \mathcal{\mathbb{R}}^n}\mathcal{L}(x,\lambda,\nu)
					\end{aligned}\]
				\vspace{-2.7mm}
			}
		\end{sstFullFrame}
	\end{centering}
\end{sstTitleBox}

\begin{proposition}[Weak Duality]
	$d(\lambda,\nu)\le f^\star,\forall\lambda\ge0,\nu\in\mathbb{R}^{h}$
\end{proposition}

\begin{definition}[Constraint qualification]
	\textbf{Slater's Condition} holds if $\exists$
	at least one
	\textbf{strictly feasible point}
	$\hat{x}{\ (h(\hat{x})=0,\ g(\hat{x})<0)}$
\end{definition}

\begin{proposition}[Strong Duality]
	If Slater's condition holds
	and OP is convex
	$\Rightarrow$
	$\exists \lambda \ge 0, \nu \in \mathbb{R}^{n_h}$ s.t. $d(\lambda,\nu)=f^\star$
\end{proposition}

\begin{sstTitleBox}[BrickRed]{\textbf{\large
			KKT  Conditions}
		\normalsize(Karush-Kuhn-Tucker)}
	\begin{theorem}[KKT Conditions]
		\begin{centering}
			If Slater's condition holds and
			(\ref{eq:dual}) is convex
			$\rightarrow$
			$x^\star \in \mathbb{R}^{n}$ is a minimizer of the primal (\ref{eq:dual})
			and $(\lambda^\star\ge0,\nu^\star)\in\mathbb{R}^{n_g}\times\mathbb{R}^{n_h}$
			is a maximizer of the dual $\Leftrightarrow$
			is equivalent to the following statements:
			\begin{sstFullFrame}[BrickRed]
				\vspace{-3mm}
				\color{white}
				\small
				\[\begin{aligned}
						\textbf{KKT-1 } & \text{\footnotesize (Stationary Lagrangian)} \hspace{-2mm} &  & \nabla_x\mathcal{L}(x^\star,\lambda^\star,\nu^\star)  =  0
						\\
						\textbf{KKT-2 } & \text{\footnotesize (primal feasibility)}                  &  & g(x^\star)\le0, h(x^\star)                            =                            0
						\\
						\textbf{KKT-3 } & \text{\footnotesize (dual feasibility)}                    &  & \lambda^\star  , \nu^\star \in \mathbb{R}^{n_h}       \ge       0
						\\
						\textbf{KKT-4 } & \text{\footnotesize (complementary}                        &  & \lambda^{\star T} g(x^\star)                          =                          0
						\\
						                & \text{\footnotesize slackness)}                            &  & \nu^{\star T} h(x^\star)                              =  0
					\end{aligned}\]
				\vspace{-4mm}
			\end{sstFullFrame}
			In addition we have:
			$\sup_{\lambda\ge0,\nu\in\mathbb{R}^{n_h}}q(\lambda,\nu)=\inf_{x\in\mathcal{C}}f(x)$
		\end{centering}
	\end{theorem}
\end{sstTitleBox}

\textbf{Remark} Without Slater,
KKT1-4 still implies $x^\star$ minimizes (\ref{eq:dual})
and $\lambda,\nu$ maximizes dual,
but the converse is no longer true.
There can be primal-minimizer/dual-maximizer not satisfy KKT.

%NOTE: Sensitivity Analysis
