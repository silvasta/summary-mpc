\section{Nominal-MPC}

\subsection{CFTOC}

\begin{sstTitleBox}[ForestGreen]{\textbf{\large
			CFTOC
		}
		Constrained Finite Time Optimal Control problem
	}
	\[
		J(x(k)) = x_N^\top P x_N + \sum_{i=0}^{N-1}x_i^\top Q x_i + u_i R u_i
	\]
	\begin{centering}
		\begin{sstFrame}[ForestGreen]
			\vspace{-1.5mm}
			\color{white}
			\[ \begin{aligned}
					J^\star (x(k)) = & \min_{U}  I_f(x_N) + \sum_{i=0}^{N-1} I(x_{i}, u_{i}) \\
				\end{aligned} \]
			\vspace{-2.5mm}
		\end{sstFrame}
		\[ \begin{aligned}
				\text{s.t} \quad & x_{i+1} = A x_{i} + Bu_{i},\ i = 0,\dots,N-1                                                  \\
				                 & x_{i} \in \mathcal{X},\ u_{i} \in \mathcal{U},\ \mathcal{X}_N \in \mathcal{X}_f, \ x_0 = x(k)
			\end{aligned} \]
		N is the time horizon and X , U, Xf are polyhedral regions
		%TODO: feasible set
	\end{centering}
	%WARN: L4.p9 combine with UCFTOC
\end{sstTitleBox}

\subsection{Transform Quadratic Cost CFTOC into QP}

\textbf{Goal}
$\min_{z\in\mathbb{R}^n}
	\textstyle\frac{1}{2}z^\top H z + q^\top z + r
	\quad\text{s.t. }Gz\leq h,\ Az = b$


\subsubsection{Substitute without substitution}

\textbf{Idea} Keep state equations as equality constraints

\textbf{Define variable} $z =
	\begin{bmatrix}
		x_1^\top & \dots x_N^\top & u_0^\top & \dots u_{N-1}^\top
	\end{bmatrix}^\top$

\textbf{Equalities} from system dynamics
$x_{i+1} = Ax_i + Bu_i$

\[
	G_{eq} =
	\begin{bmatrix}
		\begin{smallmatrix}
			\mathbb{I}& & & \\
			-A & \mathbb{I} && \\
			&  \ddots & \ddots \\
			&&-A & \mathbb{I}
		\end{smallmatrix}
		 & \vline
		\begin{smallmatrix}
			-B \\
			& \ddots \\
			& &-B
		\end{smallmatrix}
	\end{bmatrix}
	E_{eq} =
	\begin{bsmallmatrix}
		A      \\
		0      \\
		\vdots \\
		0
	\end{bsmallmatrix}
\]

\textbf{Inequalities }
$G_{in}z \leq w_{in} + E_{in}x(k)$ from
$\mathcal{X}   = \{x \mid A_x x \leq b_x\},
	\mathcal{U}   = \{u \mid A_u u \leq b_u\},
	\mathcal{X}_f = \{x \mid A_f x \leq b_f\}$
%TODO: inequality constraint order

\[
	G_{in} =
	\left[
		\begin{array}{l!{\color{RoyalBlue!50}\vrule}l}
			\noalign{\arrayrulewidth=0.1mm}\arrayrulecolor{RoyalBlue!50}
			%
			\begin{smallmatrix}
				0&&&
			\end{smallmatrix}
			 &
			\begin{smallmatrix}
				0&&&
			\end{smallmatrix}
			\\ \hline
			\begin{smallmatrix}
				A_x&&&\\
				&\ddots&\\
				&&		A_x&\\
				&&&		A_f
			\end{smallmatrix}
			 &
			\begin{smallmatrix}
				0&&&\\
				&\ddots&\\
				&&0&\\
				&&&0
			\end{smallmatrix}
			\\ \hline
			\begin{smallmatrix}
				0&&&\\
				&\ddots\\
				&&0&\\
				&&&0
			\end{smallmatrix}
			 &
			\begin{smallmatrix}
				A_u&&\\
				&A_u&\\
				&\ddots&\\
				&&A_u&\\
				&&&A_u
			\end{smallmatrix}
		\end{array}
		\right]
	%
	w_{in} =
	\begin{bmatrix}
		\noalign{\arrayrulewidth=0.1mm}\arrayrulecolor{RoyalBlue!50}
		\begin{smallmatrix}
			b_x\\
		\end{smallmatrix}
		\\ \hline
		\begin{smallmatrix}
			b_x\\
			\vdots\\
			b_x\\
			b_f\\
		\end{smallmatrix}
		\\ \hline
		\begin{smallmatrix}
			b_u\\
			b_u\\
			\vdots\\
			b_u\\
			b_u\\
		\end{smallmatrix}
	\end{bmatrix}
\]
\[
	E_{in} =
	\left[
		\begin{smallmatrix}
			-A_x \\
			0 \\
			\vdots \\
			0 \\
		\end{smallmatrix}
		\right]
\]

\textbf{Cost Matrix} $\bar{H} = \mathrm{diag}(Q,..., Q, P, R,..., R)$

\textbf{Finally the resulting quadratic optimization problem}
\[\begin{aligned}
		J^\star(x(k)) = \min_z  \left[ z^\top \ x(k)^\top \right]
		\left[\begin{smallmatrix} \bar{H} & 0 \\ 0 & Q \end{smallmatrix}\right]
		\left[ z^\top \ x(k)^\top \right]^\top \\
		\text{s.t}
		\quad G_{in}z \leq w_{in} + E_{in}x(k)
		\quad	G_{eq}z = E_{eq}x(k)
	\end{aligned}\]

\subsubsection{Substitute with substitution}

\textbf{Idea} Substitute the state equations.

\textbf{Step 1} Rewrite cost  as

\[\begin{aligned}
		J(x(k)) = & UXXXX              \\
		%TODO: step 1 substitution\\
		=         & \begin{bmatrix}
			            U^\top & x(k)^\top
		            \end{bmatrix}
		\left[\begin{smallmatrix}
				      H & F^\top \\
				      F & Y
			      \end{smallmatrix}\right]
		\begin{bmatrix}
			U^\top & x(k)^\top
		\end{bmatrix}^\top
	\end{aligned} \]

\textbf{Step 2} Rewrite constraints compaclty as $GU\le w+Ex(k)$
%TODO: G,U,...

\textbf{Step 3} Rewrite constrained problem as
%HACK: box?
\[\begin{aligned}
		J^\star(x(k)) = \min_U
		                       & \begin{bmatrix}
			                         U^\top & x(k)^\top
		                         \end{bmatrix}
		\left[\begin{smallmatrix}
				      H & F^\top \\
				      F & Y
			      \end{smallmatrix}\right]
		\begin{bmatrix}
			U^\top & x(k)^\top
		\end{bmatrix}^\top                          \\
		\mathrm{subj. \ to \ } & GU \leq w + Ex(k)
	\end{aligned} \]

%HACK: L_p norm, and transform cftoc to LP

%HACK: Receiding horizon control Notation
%
\subsection{Invariance}

\begin{definition}[Positively Invariant Set $\mathcal{O}$]
	For an autonomous or closed-loop system,
	the set $\mathcal{O}$ is positively invariant if:
	\[
		x(k)\in\mathcal{O}\Rightarrow
		x(k+1) \in \mathcal{O},
		\quad \forall k \in \{0,1,\dots\}
	\]
\end{definition}

\begin{definition}[Maximal Positively Invariant Set $\mathcal{O}_\infty$]
	A set that contains all $\mathcal{O}$
	is the maximal positively invariant set
	$\mathcal{O}_\infty \subset \mathcal{X}$
\end{definition}

\begin{definition}[Pre-Sets]
	The set of states that
	in the dynamic system $x(k+1) = g(x(k))$
	in one time step evolves into the target set $\mathcal{S}$
	is the \textbf{pre-set} of $\mathcal{S}$
	$\Rightarrow \pre(\mathcal{S}) := \{x\mid g(x)\in \mathcal{S}\}$
\end{definition}
\begin{theorem}[Geometric condition for invariance]
	Set $\mathcal{O}$ is positively invariant set iff
	$\mathcal{O}\subseteq\pre(\mathcal{O})
		\Leftrightarrow
		\pre(\mathcal{O})\cap\mathcal{O} = \mathcal{O}$
\end{theorem}
\begin{proof}
	\textbf{Necessary} if
	$\mathcal{O} \nsubseteq  \mathrm{pre}(\mathcal{O})$,
	then $\exists\bar{x} \in \mathcal{O}$
	s.t $\bar{x} \notin \mathrm{pre}(\mathcal{O})$
	$\leadsto \bar{x}\in\mathcal{O},
		\bar{x}\notin\mathrm{pre}(\mathcal{O})$,
	thus $\mathcal{O}$ not positively invariant

	\textbf{Sufficient} if
	$\mathcal{O}$ not pos invar set,
	then $\exists \bar{x}\in\mathcal{O}$
	s.t $g(\bar{x}) \notin\mathcal{O}$
	$\leadsto \bar{x}\in\mathcal{O},
		\bar{x}\notin\mathrm{pre}(\mathcal{O})$
	thus $\mathcal{O}\notin \mathrm{pre}(\mathcal{O})$
\end{proof}


Computing Invariant Sets

\begin{minipage}[b]{0.495\linewidth}
	\begin{sstTitleBox}[ForestGreen]{\center\textbf
			Pre-Set Computation}
		System with constraints

		$x(k+1) = Ax(k) +Bu(k)$

		$u(k)\in\mathcal{U}:=\{u|Gu\le g\}$

		and set
		$\mathcal{S}: = \{x|Fx\leq f\}$

		$ 			\mathrm{pre}(S):  =  \{x \mid Ax \in S\}  $
		$ 			=  \{ x \mid FAx \leq f\}                 $
	\end{sstTitleBox}
\end{minipage}
\begin{minipage}[b]{0.49\linewidth}
	\begin{sstTitleBox}[ForestGreen]{\center\textbf
			Conceptual Algorithm}
		\begin{sstOnlyFrame}[ForestGreen]
			\begin{algorithmic}
				\State first line
				\State $\Omega_0 \leftarrow \mathcal{X}$
				\Loop
				\State $\Omega_{i+1} \leftarrow \mathrm{pre}(\Omega_i)\cap\Omega_i$
				\If{$\Omega_{i+1}=\Omega_i$}
				\State\Return $\mathcal{O}_\infty = \Omega_i$
				\EndIf
				\EndLoop
			\end{algorithmic}
		\end{sstOnlyFrame}
		(Same but much harder for control invariat sets)
	\end{sstTitleBox}
	%FIX: set proper Algorithm frame
\end{minipage}

\subsection{Control Invariance}


\begin{definition}[Control Invariant Set]
	$\mathcal{C} \subseteq \mathcal{X}$ control invariant if
	$$x(k) \in \mathcal{C} \Rightarrow
		\ \exists u(k) \in \mathcal{U} \text{ s.t }
		g(x(k),u(k))\in\mathcal{C} \ \forall k$$
\end{definition}

\begin{definition}[Maximal Control Invariant Set $\mathcal{C}_\infty$]
	A set that contains all $\mathcal{C}$
	is the maximal positively invariant set
	$\mathcal{C}_\infty \subset \mathcal{X}$

	\textbf{Intuition} For all states in $\mathcal{C}_\infty$
	exists control law s.t constraints are never violated
	$\leadsto$ \textbf{The best any controller could ever do}
\end{definition}

\textbf{Pre-set}
$\pre(\mathcal{S}):=\{x \mid \exists u\in\mathcal{U}
	\text{ s.t } g(x,u) \in \mathcal{S}\}$

Set $\mathcal{C}$ is control invariant iff:
$\mathcal{C}\subseteq\pre(\mathcal{C})
	\Leftrightarrow
	\pre(\mathcal{C})\cap\mathcal{C} = \mathcal{C}$
%WARN: pre set controll invariance, L5.p43

\begin{sstTitleBox}[BrickRed]{\center\textbf%{\large
		Control Law from Control Invariant Set
	}%}

	\begin{sstOnlyFrame}[BrickRed]

		\begin{centering}
			Let	$\mathcal{C}$ control invariant set
			for $x(k+1)=g(x(k),u(k))$

			Control law $\kappa(x(k))$ will \textbf{guarantee}
			that system satisfies constraints \textbf{for all time} if:
			$g(x,\kappa(x)) \in \mathcal{C} \ \forall x \in \mathcal{C}$

			We can use this fact to \textbf{synthesize}
			control law $\kappa$

			with $f$ as any function (including $f(x,u)=0)$

		\end{centering}
		\[
			\kappa(x):=\argmin\{f(x,u) \mid g(x,u)\in\mathcal{C}\}
		\]

		Does not ensure that system will converge

		Difficult because calculating control invariant sets is hard

		\textbf{MPC} implicitly describes $\mathcal{C}$
		s.t easy to represent/compute
	\end{sstOnlyFrame}
\end{sstTitleBox}

\begin{theorem}{Minkowski-Weyl}

	The following statements are equivalent
	for $\mathcal{P}\subseteq \mathbb{R}^d$
	\begin{itemize}[leftmargin=1em]
		\item $\mathcal{P}$ is a polytope and there exists
		      $A, b$ s.t $\mathcal{P} = \{x \mid Ax \leq b\}$
		\item $\mathcal{P}$ finitely generated,
		      $\exists$ finite set $\{v_i\}$ s.t
		      $\mathcal{P}=\text{co}(\{v_1,...,v_s\})$
	\end{itemize}
\end{theorem}


\begin{sstTitleBox}[ForestGreen]{\textbf{\large
			MOST COMMON Polytopic
		}
		% small text
	}

	%WARN: Most Common Polytopic constraints? L5.p61

	\[\begin{aligned}
			1
		\end{aligned}\]

	%TODO: Intersection of Polytopes Compute Polytopic Invariant Set, example e5 Subset test
\end{sstTitleBox}

\begin{lemma}{Invariant Sets from Lyapunov Functions}
	%WARN: move to Lyapunov?

	If $V:\mathbb{R}^n \to \mathbb{R}$ is a Lyapunov function
	for $x(k+1) = g(x(k))$, then
	$Y := \{x \mid V(x) \leq \alpha\}$
	is an invariant set for all $\alpha \geq 0$

	\begin{proof}
		Lyapunov property $V(g(x)) - V(x) < 0$
		implies that once $V(x(k))\leq \alpha$,
		$V(x(j))<\alpha$,
		$\forall\ j\ge k \rightarrow$ Invariance
	\end{proof}

	\textbf{Example System} for $x(k+1) = Ax(k)$ with $P\succ 0$ that satisfies $A^\top P A - P \prec 0$ $\leadsto$ then $V(x(k)) = x(k)^\top P x(k)$ is Lyap. function
	%TODO: rewrite example l5.p71

	Goal -- find largest $\alpha$ s.t set $Y_\alpha \in \mathcal{X}$ \\
	$\qquad Y_\alpha := \{x \mid x^\top P x \leq \alpha\}\subset \mathcal{X} := \{x \mid Fx\leq f\}$ \\
	Equivalent to $\qquad \max_\alpha \alpha \quad \mathrm{subj.\ to}\ h_{Y_\alpha}(F_i) \leq f_i \ \forall i \in \{1\dots n\}$
\end{lemma}

%HACK: Maximum Elipsoid set

\subsection{Feasibility and Stability}

%TODO: move to proper place
The first reason is that re-optimization provides robustness to any noise or modeling errors,
while the second is that the solution at time k = 0 is sub-optimal because it is over a finite
horizon. Re-optimizing can provide a control law with better performance.

\begin{sstTitleBox}[ForestGreen]{\center\textbf{\large
			MPC Mathematical Formulation
		}
		% small text
	}

	\begin{sstFrame}[ForestGreen]
		\small
		\color{white}
		\vspace{-1mm}
		\[
			%HACK: Make u instead j
			J^\star(x(k))= \min\sum_{i=0}^{N-1}
			x_i^\top Q x_i + u_i^\top R u_i\]
		\vspace{-2mm}
	\end{sstFrame}

	\[\begin{aligned}
			\mathrm{subj.\ to }\   x_{i+1} & = Ax_i + Bu_i \\
			x_0                            & = x(k)
			\quad x_i \in \mathcal{X},
			\quad u_i \in \mathcal{U}
		\end{aligned}\]


\end{sstTitleBox}


%TODO: Lyapunov Stability Theorem
%
%HACK: MPC Proof 1,2


\begin{sstTitleBox}
	{	\subsubsection{Stability of MPC - Main Result}}
	\begin{theorem}
		The closed-loop system under the MPC control law $u_0^\star(x)$
		is asymptotically stable and the set $\mathcal{X}_f$
		is positive invariant for the system
		$x(k+1) = Ax(k) + Bu_0^\star(x(k))$
		under the following assumptions:

		1. Stage cost is positive definite, i.e. it is strictly positive and only zero at
		the origin

		2. Terminal set is \textbf{invariant}
		under the local control law $\kappa_f(x_i)$:

		\[
			x_{i+1} = Ax_i + B\kappa_f(x_i) \in \mathcal{X}_f
			\text\forall x_i \in \mathcal{X}_f
		\]

		All state and input \textbf{constraints are satisfied} in $\mathcal{X}_f$:

		\[
			\mathcal{X}_f \in X, \kappa_f(x_i) \in U
			\text\forall x_i \in \mathcal{X}_f
		\]

		3. Terminal cost is a continuous \textbf{Lyapunov function}
		in the terminal set $\mathcal{X}_f$ and satisfies:
		\[
			I_f(x_{i+1}) - I_f(x_i) \leq
			- I(x_i, \kappa_f(x_i)) \quad
			\forall x_i \in \mathcal{X}_f
		\]
	\end{theorem}
\end{sstTitleBox}

%TODO: choice of terminal set

%TODO: Extension to nonlinear MPC

\begin{sstFrame}[ForestGreen!20]
	% \center

	\textbf{Finite-horizon MPC may not satisfy constraints for all time!}

	\textbf{Finite-horizon MPC may not be stable!}

	-	An infinite-horizon provides stability and invariance.

	-	Infinite-horizon \textit{faked} by forcing final state
	into an invariant set for which there exists
	invariance-inducing controller,
	whose infinite-horizon cost can be expressed in closed-form.

	-	Extends to non-linear systems, but compute sets is difficult!
\end{sstFrame}
