\section{Nominal-MPC}


\subsection{CFTOC}

\begin{sstTitleBox}[ForestGreen]{\textbf{\large
			CFTOC
		}
		Constrained Finite Time Optimal Control problem
	}
	\[
		J(x(k)) = x_N^\top P x_N + \sum_{i=0}^{N-1}x_i^\top Q x_i + u_i R u_i
	\]
	\begin{centering}
		\begin{sstFrame}[ForestGreen]
			\vspace{-1.5mm}
			\color{white}
			\[ \begin{aligned}
					J^\star (x(k)) = & \min_{U}  I_f(x_N) + \sum_{i=0}^{N-1} I(x_{i}, u_{i}) \\
				\end{aligned} \]
			\vspace{-2.5mm}
		\end{sstFrame}
		\[ \begin{aligned}
				\text{s.t} \quad & x_{i+1} = A x_{i} + Bu_{i},\ i = 0,\dots,N-1                                                  \\
				                 & x_{i} \in \mathcal{X},\ u_{i} \in \mathcal{U},\ \mathcal{X}_N \in \mathcal{X}_f, \ x_0 = x(k)
			\end{aligned} \]
		N is the time horizon and X , U, Xf are polyhedral regions
		%TODO: feasible set
	\end{centering}
\end{sstTitleBox}

%TODO: Without Substitution

%TODO: With Substitution

%TODO: Rest of CFTOC
%

\subsection{Invariance}

%HACK: Definitions

%HACK: Invariance + feedback law
%HACK: Control Invariance
%HACK: Polytopes, definition r6 sum
%HACK: Ellipsoid 

\subsection{Feasibility and Stability}

%TODO: move to proper place
The first reason is that re-optimization provides robustness to any noise or modeling errors,
while the second is that the solution at time k = 0 is sub-optimal because it is over a finite
horizon. Re-optimizing can provide a control law with better performance.

\begin{sstTitleBox}[ForestGreen]{\center\textbf{\large
			MPC Mathematical Formulation
		}
		% small text
	}

	\begin{sstFrame}[ForestGreen]
		\small
		\color{white}
		\vspace{-1mm}
		\[
			%HACK: Make u instead j
			J^\star(x(k))= \min\sum_{i=0}^{N-1}
			x_i^\top Q x_i + u_i^\top R u_i\]
		\vspace{-2mm}
	\end{sstFrame}

	\[\begin{aligned}
			\mathrm{subj.\ to }\   x_{i+1} & = Ax_i + Bu_i \\
			x_0                            & = x(k)
			\quad x_i \in \mathcal{X},
			\quad u_i \in \mathcal{U}
		\end{aligned}\]


\end{sstTitleBox}


%TODO: Lyapunov Stability Theorem
%
%HACK: MPC Proof 1,2


\begin{sstTitleBox}
	{	\subsubsection{Stability of MPC - Main Result}}
	\begin{theorem}
		The closed-loop system under the MPC control law $u_0^\star(x)$
		is asymptotically stable and the set $\mathcal{X}_f$
		is positive invariant for the system
		$x(k+1) = Ax(k) + Bu_0^\star(x(k))$
		under the following assumptions:

		1. Stage cost is positive definite, i.e. it is strictly positive and only zero at
		the origin

		2. Terminal set is \textbf{invariant}
		under the local control law $\kappa_f(x_i)$:

		\[
			x_{i+1} = Ax_i + B\kappa_f(x_i) \in \mathcal{X}_f
			\text\forall x_i \in \mathcal{X}_f
		\]

		All state and input \textbf{constraints are satisfied} in $\mathcal{X}_f$:

		\[
			\mathcal{X}_f \in X, \kappa_f(x_i) \in U
			\text\forall x_i \in \mathcal{X}_f
		\]

		3. Terminal cost is a continuous \textbf{Lyapunov function}
		in the terminal set $\mathcal{X}_f$ and satisfies:
		\[
			I_f(x_{i+1}) - I_f(x_i) \leq
			- I(x_i, \kappa_f(x_i)) \quad
			\forall x_i \in \mathcal{X}_f
		\]
	\end{theorem}
\end{sstTitleBox}

%TODO: choice of terminal set

%TODO: Extension to nonlinear MPC

\begin{sstFrame}[ForestGreen!20]
	% \center

	\textbf{Finite-horizon MPC may not satisfy constraints for all time!}

	\textbf{Finite-horizon MPC may not be stable!}

	-	An infinite-horizon provides stability and invariance.

	-	Infinite-horizon \textit{faked} by forcing final state
	into an invariant set for which there exists
	invariance-inducing controller,
	whose infinite-horizon cost can be expressed in closed-form.

	-	Extends to non-linear systems, but compute sets is difficult!
\end{sstFrame}
